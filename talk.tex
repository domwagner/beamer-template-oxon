\documentclass{beamer}

\newif\ifdraft

%%%%%%%%%%%%%%%%%%%%%%%%%%

\newcommand\defeq{\coloneqq}

%%%%%%%%%%%%%%%%%%%%%%%%%% 

\usepackage[british,UKenglish]{babel}
\usepackage[T1]{fontenc}
\usepackage{stmaryrd}
\usepackage{mathtools}
\usepackage{amsmath}
\usepackage{amssymb}
\usepackage{amsfonts}
\usepackage{amsthm}
\usepackage{tabularx}

\usepackage{listings}
\lstset{language=haskell, keywordstyle={\bfseries\color{oxblue}},morekeywords={rec}}
\usepackage{stackengine}

\usepackage{tikz}
\usetikzlibrary{positioning}
\usetikzlibrary{arrows.meta, decorations.pathreplacing, shadows}

\newcommand\sinti[2]{{#1}\llbracket#2\rrbracket}


\DeclareMathOperator{\dom}{dom}


\usepackage{graphbox}
\usepackage{mdframed}

\usepackage{style/beamerthemeoxon}

\usepackage{tikz}
\usetikzlibrary{calc, positioning, shapes.arrows,matrix,fit,patterns}
\usepackage{tabularx}
\usepackage{pifont}

\newcommand{\cmark}{\color{oxgreen}\ding{51}}%
\newcommand{\xmark}{\color{oxdarkerred}\ding{55}}

\usepackage{multirow}
\usepackage{tikz-cd}

\newcommand{\subheadline}[1]{\bigskip\textbf{\large\color{oxlighterblue}#1}}

\newcounter{partc}

\newcommand{\partframe}[1]{
  \begin{frame}[plain]
    \begin{columns}
      \begin{column}{\paperwidth}
        \begin{mdframed}[backgroundcolor=oxlighterblue,fontcolor=white,
          userdefinedwidth=\paperwidth,
          innertopmargin=10pt,innerbottommargin=10pt,
          innerleftmargin=2pt,innerrightmargin=2pt]

          \centering
          \stepcounter{partc}
          \Huge Part \Roman{partc}:

          \textbf{#1}
        \end{mdframed} 
      \end{column}
    \end{columns}
  \end{frame}
}

\tikzstyle{arr}=[->,color=oxlighterblue]
\tikzstyle{ann}=[text=oxlighterblue,anchor=base]
\tikzstyle{box}=[fill=oxlightblue,draw=oxblue,inner sep=5pt,very
      thick,anchor=base]
\tikzstyle{greybox}=[fill=black!5,draw=black!75,inner sep=5pt,very
      thick,anchor=base]
      \tikzstyle{e}=[anchor=base]
      \tikzstyle{frame}=[fill=oxpink,anchor=base,very
  thick,dashed,ellipse,inner sep=1pt]
  \tikzstyle{highlight}=[fill=oxpink,draw=oxred,anchor=base,very
  thick,inner sep=5pt]
  \tikzstyle{harr}=[->,color=oxred]

  

\tikzset{
    invisible/.style={opacity=0},
    visible on/.style={alt={#1{}{invisible}}},
    alt/.code args={<#1>#2#3}{%
      \alt<#1>{\pgfkeysalso{#2}}{\pgfkeysalso{#3}}%
  }
}

\newtheorem{proposition}[theorem]{Proposition}
\usepackage{graphbox}
\usepackage{bussproofs}
\usepackage{mdframed}

\usepackage{style/beamerthemeoxon}

\usepackage{tikz}
\usetikzlibrary{calc, positioning, shapes.arrows,matrix,fit,patterns}
\usepackage{tabularx}
\usepackage{pifont}

\newcommand{\cmark}{\color{oxgreen2}\ding{51}}%
\newcommand{\xmark}{\color{oxred}\ding{55}}

\usepackage{multirow}
\usepackage{tikz-cd}

\newcommand{\inferbintalk}[5]{\bigskip\begin{minipage}{12ex}{\color{oxblue}\bfseries #1}\end{minipage} {
    \begin{minipage}{10ex}{$\infer{#4}{#2 &&& #3}$}\end{minipage}
    \par\medskip\noindent #5}}
\newcommand{\inferuntalk}[4]{\bigskip\begin{minipage}{12ex}{\color{oxblue}\bfseries #1}\end{minipage} {
    \begin{minipage}{10ex}{$\infer{#3}{#2}$}\end{minipage}
    \par\bigskip\noindent #4}}

\newcommand{\subheadline}[1]{\bigskip\textbf{\large\color{lighterblue}
    #1}}


\newcounter{partc}

\newcommand{\partframe}[1]{
  \begin{frame}[plain]
    \begin{columns}
      \begin{column}{\paperwidth}
        \begin{mdframed}[backgroundcolor=lighterblue,fontcolor=white,
          userdefinedwidth=\paperwidth,
          innertopmargin=10pt,innerbottommargin=10pt,
          innerleftmargin=2pt,innerrightmargin=2pt]
      
          \centering
          \stepcounter{partc}
          \Huge Part \Roman{partc}:

          \textbf{#1}
        \end{mdframed} 
      \end{column}
    \end{columns}
  \end{frame}
}

\tikzstyle{arr}=[->,color=lighterblue]
\tikzstyle{ann}=[text=lighterblue,anchor=base]
\tikzstyle{box}=[fill=lightblue,draw=oxblue,inner sep=5pt,very
      thick,anchor=base]
\tikzstyle{greybox}=[fill=black!5,draw=black!75,inner sep=5pt,very
      thick,anchor=base]
      \tikzstyle{e}=[anchor=base]
      \tikzstyle{frame}=[fill=oxpink,anchor=base,very
  thick,dashed,ellipse,inner sep=1pt]
  \tikzstyle{highlight}=[fill=oxpink,draw=oxred2,anchor=base,very
  thick,inner sep=5pt]
  \tikzstyle{harr}=[->,color=oxred2]

  

\tikzset{
    invisible/.style={opacity=0},
    visible on/.style={alt={#1{}{invisible}}},
    alt/.code args={<#1>#2#3}{%
      \alt<#1>{\pgfkeysalso{#2}}{\pgfkeysalso{#3}}%
  }
}


\newtheorem{proposition}[theorem]{Proposition}


\begin{document}

\title[HoCHC]{
  \textbf{HoCHC: A Refutationally Complete and\\
    Semantically Invariant System of\\
    Higher-order Logic Modulo Theories}
}
\author[L.~Ong, \textbf{\underline{D.~Wagner}}]{Luke Ong\and
  \textbf{ \underline{Dominik Wagner}}}
\institute{\includegraphics[width=2.7cm]{logo2.png}}
\date[LICS 2019]{LICS 2019% \\\vspace{0.2cm}24th
  % June 2019
}

\frame{\titlepage}

\section{Introduction}

\begin{frame}
  % \frametitle{Motivation}
  \onslide<3->{
    \hspace{-0.8cm}\tikz{
      \draw [thick, decoration={brace, mirror,raise=0.5cm,
        amplitude=8pt}, decorate] (0,0) -- (2,0) 
      node [pos=0.5,anchor=north,yshift=0.15cm] {\Large\emph{1st-order}};
    }
  }
  
  \textit{\LARGE\color{lighterblue}
    ``Constrained Horn Clauses provide a \alert<2->{suitable} basis for automatic program verification''
  }

  \onslide<3->{
    \hspace{3.4cm}\tikz[baseline]{
      \draw [thick, decoration={brace,raise=0.5cm,
        amplitude=8pt}, decorate] (0,-0.5) -- (2,-0.5) 
      node [pos=0.5,anchor=north,yshift=0.6cm] {\Large\emph{imperative}};
    }}\hfill [Bj{\o}rner et al., 2015]

  \onslide<2->{
    \setbeamertemplate{itemize item}[triangle]
    \setbeamercolor{itemize item}{fg=oxblue}
    \vspace{1cm}
    {
      \begin{itemize}
      \item separation of concerns
      \item good \emph{algorithmic} properties: semi-decidable, highly efficient solvers
      \end{itemize}
    }
  }

  \vspace{0.5cm}
  \onslide<4->{{[Cathcart Burn, Ong \& Ramsay; POPL'18]:}

    {\large
    \textbf{\textit{\color{lighterblue}extend approach to \emph{higher-orders}}}}
  }

\end{frame}

% \begin{frame}
  
%   {\Large
%     \cite{BOR18}: \color{lighterblue}extend approach to higher-orders
%   }

% \end{frame}

\begin{frame}[fragile]

  \vspace*{3mm}
  {\large 
    \only<2->{\lstset{emph={f},emphstyle={\color{oxred2}}}}
    \begin{lstlisting}[escapeinside={(*}{*)}]
let add   x y = x + y(*\pause*)
let twice f x = f (f x)(*\pause*)
in  (*$\lambda$*)x. assert (x >= 1 ->
                    (twice (add x) 0) > x)
\end{lstlisting}
}
\pause
  \vspace{0.5cm}

  \newcolumntype{R}{>{\raggedleft\arraybackslash}X}
  \newcolumntype{L}{>{\raggedright\arraybackslash}X}
  \renewcommand\tabularxcolumn[1]{m{#1}}
  \begin{tabularx}{\textwidth}{ RcL }
    \begin{tikzpicture}
      \coordinate (b) at (0,0);
      \coordinate (a) at (0,1);
      \draw[->, >=latex, oxblue, line width=5pt]   (a) to node[white]{} (b) ;
    \end{tikzpicture} &
    (over-)approximate graph of functions
    &\begin{tikzpicture}
      \coordinate (b) at (0,0);
      \coordinate (a) at (0,1);
      \draw[->, >=latex, oxblue, line width=5pt]   (a) to node[white]{} (b) ;
    \end{tikzpicture}
  \end{tabularx}
  \pause

  \Large
  \vspace{-0.2cm}
    \[
      \begin{array}{l}
        {\scriptstyle\forall x, y, z\ldotp\big(}z = x + y \to \Add\,  x \, y \, z {\scriptstyle\big)} \pause\\[4pt]
        {\scriptstyle\forall f, x,z\ldotp \big(} \exists y\ldotp
        (f \, x \, y)\land (f\,y\,z) \to \Twice \, f \, x\,z {\scriptstyle\big)} \pause\\[4pt]
        \onslide<5->{{\scriptstyle\forall x,z\ldotp \big (} x \geq 1 \wedge \Twice \, (\Add\,x) \, 0\,z\; \to \; z > x {\scriptstyle\big)}}
      \end{array}
    \]
    
\end{frame}


% \begin{frame}

%   \tikzstyle{every picture}+=[remember picture]

%   \vspace*{5mm}
%   \begin{center}
%     \begin{tikzpicture}
%       \node [text=lighterblue] (HoCHC) {\LARGE\textbf{HoCHC for verification}};
%       \onslide<2->{\node [box,align=center,text width=3.5cm,below
%         right=7mm and -24mm of HoCHC] (TA) {\large (A)TA higher-order model checking};}
%       \node [box,align=center,text width=2.7cm,below left=7mm and -24mm of HoCHC] (dir) {\large direct encoding of safety};
      
%       \onslide<3->{\node [below=3mm of TA] (TAexp) {\emph{$\mathcal
%             L(\mathcal A)\ni \llbracket\mathcal G\rrbracket$}};}
      
%       \path<2->[<-, >=latex, color=lighterblue, line width=4pt] (HoCHC) edge (TA);
%       \path[<-, >=latex, color=lighterblue, line width=4pt] (HoCHC) edge (dir);
%     \end{tikzpicture}
%   \end{center}

%   \vspace*{5mm}
%   \hspace*{2cm}
%   \onslide<2-3>{\tikz[baseline]{\node [ann,text width=2.6cm] (TAann) {(alternating)\\trivial automata};}}
%   \hspace*{1cm}
%   \tikz[baseline]{\node<3-> [ann,text width=2.8cm]
%     (HoRSann) {higher-order\\recursion scheme};}
%   \vspace*{-1mm}
%   \onslide<3->{
%     \hfill
%     \color{oxbrown!60!black}
%   \begin{figure}
%     \begin{minipage}{0.1\textwidth}
%       \hfill
%     \end{minipage}
%     \begin{minipage}{0.3\textwidth}
%       \tikz[baseline]{\node[anchor=base] (HoRS1) {$\mathcal G$};}defined by
%       {\small
%         \begin{align*}
%           S&=F\,b\,c\\
%           F\,f\,x&=a\,(f\,x)\,(F\,f\,(f\,x))  
%         \end{align*}
%       }
%     \end{minipage}
%     \begin{minipage}{0.2\textwidth}
%       \hfill$\llbracket\mathcal G\rrbracket=$
%     \end{minipage}
%     \begin{minipage}{0.3\textwidth}
%       \begin{tikzpicture}[node distance=8mm,scale=0.9, every node/.style={scale=0.9}]
%         \tikzstyle{state}=[text=oxred]
        
%         \node (r) {a};
%         \onslide<-3>{\node [state,above right=-10pt of r] {$q_0$};}
%         \node [below left of=r] (b1) {b};
%         \onslide<-3>{\node [state,above left=-10pt of b1] {$q_0$};}
%         \node [below of=b1] (c1) {c};
%         \onslide<-3>{\node [state,above left=-10pt of c1] {$q_1$};}
%         \node [right=1.1cm of b1] (a) {a};
%         \onslide<-3>{\node [state,above right=-10pt of a] {$q_0$};}
%         \node [below left of=a] (b2) {b};
%         \onslide<-3>{\node [state,above left=-10pt of b2] {$q_0$};}
%         \node [below right of=a] (dots) {\vdots};
%         \node [below of=b2] (b3) {b};
%         \onslide<-3>{\node [state,above left=-10pt of b3] {{\Large $\lightning$}$q_1$};}
%         \node [below of=b3] (c2) {c};

%         \path (r) edge  (b1)
%         edge (a)
%         (b1) edge (c1)
%         (a) edge (b2)
%         edge (dots)
%         (b2) edge (b3)
%         (b3) edge (c2);
%       \end{tikzpicture}
%     \end{minipage}
%   \end{figure}
%   }

%   \begin{tikzpicture}[overlay]
%     \path<2-3>[->,arr] (TAann) edge [bend left=20] ($ (TA.south west) + (0.3,0.5) $);
%     \path<3>[->,arr] ($ (TAann.east) - (0.5,0) $) edge [bend right] ($ (TAexp.south west) + (0.7,0) $)
%      (HoRSann.north) edge [bend right] ($ (TAexp.south west) + (1.7,0) $);
%   \end{tikzpicture}  
% \end{frame}

\begin{frame}
  
  {\LARGE
    %\color{foo}
    \color{lighterblue}
    \textit{Is higher-order (Horn) logic modulo theories a sensible
    \emph{algorithmic} approach to verification?}
  
  % \vspace{0.5cm}\pause
  
  % \textit{\textbf{Is it well-founded?}}
}
  
\end{frame}

\begin{frame}

  \tikzstyle{every picture}+=[remember picture]
  \only<5->{\tikzstyle{strthr}=[thick,draw=oxred2,cross
    out,anchor=base,inner sep=2pt]}
  \only<-4>{\tikzstyle{strthr}=[anchor=base,inner sep=2pt]}
  \only<5->{\tikzstyle{fail}=[ellipse,inner
    sep=0pt,pattern=crosshatch,pattern color=oxred,anchor=base]}
  \only<-4>{\tikzstyle{fail}=[anchor=base, inner sep=0pt]}

  
  % \onslide<3->{\hspace*{82mm}\tikz{\node[ann,text width=3cm] (folc) {\textit{$\cong$ FoL + comprehension axioms}}}}
  {
    \bfseries\color{oxblue}
    \begin{table}
      \centering
      \hspace*{-6mm}
      \renewcommand\arraystretch{1.5}
      \begin{tabular}{l|cccc}
        &\multirow{2}{*}{1st-order logic}
        \onslide<2->{&
                    \multicolumn{2}{c}{
                    \tikz[baseline]{
                    \node[strthr] (hol)
                    {higher-order logic};}}
        \\\cline{3-4}}
        &&\onslide<2->{\tikz[baseline]{\node[anchor=base ,inner sep=0pt]
           (std){standard};}}
                     & \onslide<3->{
                       \tikz[baseline]{
                           \node[anchor=base,inner sep=0pt] (Henkin) {Henkin};
                       }
                       % \tikz[overlay]{
                       % \path[->] ($ (folc.south east) + (-0.3,0.3) $)
                       % edge [bend left] (Henkin.north east) ;
                       % }
                       }
        \\\hline
        complete proof systems&\cmark&
                                       \tikz[baseline]{
                                       \node<2->[fail] (proof)
                                       {\xmark};}&\onslide<3->{\cmark}\\
        semi-decidable&\cmark&
                               \tikz[baseline]{
                                       \node<2->[fail] (semi)
                                       {\xmark};}&\onslide<3->{\cmark}\\\pause\pause
        1st-order translation&--&
                                       \tikz[baseline]{
                                       \node[fail] (fol)
                                       {\xmark};}&\cmark\\\pause
        intuitive &\cmark&\cmark&\xmark
      \end{tabular}
    \end{table}

    \begin{tikzpicture}[overlay]
      \only<5->{
        \node[anchor=base,above=0cm of hol]
        {\emph{\Large\textbf{HoCHC}}};
        \node[anchor=base,right=-0.1cm of proof]
        {\cmark};
        \node[anchor=base,right=-0.1cm of semi]
        {\cmark};
        \node[anchor=base,right=-0.1cm of fol]
        {\cmark};
      }
      \only<6->{\node[anchor=base,right=-0.4mm of std]
        {\emph{$\Leftrightarrow$}};}
    \end{tikzpicture}
    
  }

\end{frame}

\begin{frame}
  \frametitle{Contributions}
  \begin{itemize}
  \item A \emph{simple} resolution proof system for HoCHC
    \begin{itemize}
    \item Completeness even for \emph{standard} semantics
    \item HoCHC is \emph{semi-decidable} and compact
    \end{itemize}
  \item Semantic invariance\pause
  \item Canonical model property
  \item 1-st order translation (complete for \emph{standard}
    semantics)
  \item \emph{Decidable} fragments
  \end{itemize}

  \pause
  \subheadline{This talk:}
  \begin{itemize}
  \item Canonical model property 
  \item Resolution proof system and its completeness
  \item Semantic invariance
  \end{itemize}
\end{frame}

\section{HoCHC}
\partframe{HoCHC}

\begin{frame}
  \frametitle{Syntactic Features}
  \tikzstyle{every picture}+=[remember picture]

  \only<2>{\tikzstyle{bg}=[fill=oxgold!25,anchor=base,inner sep=2pt,very
    thick,dashed]}
  \only<1,3->{\tikzstyle{bg}=[anchor=base,inner sep=2pt,very
    thick,dashed]}

  \only<3>{\tikzstyle{hol}=[fill=oxgreen3,anchor=base,inner sep=2pt]}
  \only<1,2,4->{\tikzstyle{hol}=[anchor=base,inner sep=2pt]}

  \only<5->{\tikzstyle{disvars}=[fill=oxpink,anchor=base,inner sep=2pt]}
  \only<1-4>{\tikzstyle{disvars}=[fill=white,anchor=base,inner sep=2pt]}


  \vspace{-0.1cm}
  signatures $\tikz[baseline]{
      \node[anchor=base] (sigs1)
      {$\Sigma$};}\subseteq \tikz[baseline]{
      \node[anchor=base] (sigs2)
      {$\Sigma'$};}$
    \hfill
    \onslide<5->{\tikz\node[ann] (disvars1)
      {\emph{distinct} \color{lighterblue}variables}; \hspace*{8mm}}

  \onslide<2-3>{
    \hspace{-0.5cm}\tikz[baseline]\node[ann,bg] (sigsann1){background theory};
    \hspace{0.2cm}\tikz[baseline]\node[ann] (sigsann2)  {relational extension}; 
  }
  \onslide<4>{\hfill\tikz[baseline]\node[ann] (def)  {definite clauses}; \hspace*{8mm}} 

  \vspace*{-0.4cm}
  

    
  \begin{minipage}{.6\textwidth}
    \begin{align*}
      \tikz[baseline]{\node[bg]{$\neg (z=x+y)$};}
      \lor \Add\tikz[baseline]{
      \node[disvars] (disvars2)
      {$x\,y\,z$};}\\
      \tikz[baseline]{\node[hol] (hoh)
      {$\neg(f\,x\,y)$};} \lor
      \tikz[baseline]{\node[hol] (hoh)
      {$\neg(f\,y\,z)$};} \lor
      \Twice\tikz[baseline]{
      \node[disvars] (disvars4)
      {$f\,x\,z$};} \\
      \tikz[baseline]{\node[bg]{$\neg (x\geq 1)$};}
      \lor\neg\Twice\,(\Add\,x)\,0\,z\lor
      \tikz[baseline]{\node[bg] (goal2) {$\neg (z\leq x)$};}
    \end{align*}
  \end{minipage}%
  \begin{minipage}{0.4\textwidth}
  \end{minipage}
  
  \onslide<4>{\hfill\tikz[baseline]\node[ann] (goal)  {goal clause};\hspace*{8mm}}
  \begin{itemize}
  \item<3-> only \emph{relational} higher-order types 
  \item<5-> positive literals are \emph{definitional}
  \item<6-> no logical symbols in atoms: \tikz[baseline]{
      \node[thick,draw=oxred2,cross out,anchor=base,inner sep=0pt]
      {$R\,\neg$};}
  \item<6-> in paper: + $\lambda$-abstractions
  \end{itemize}

  

  \begin{tikzpicture}[overlay]
    \path[arr]<2-3> (sigsann1) edge [bend left=20] (sigs1.south west)
    (sigsann2.north) edge [bend right] (sigs2);
    \path[arr]<4> (def.south) edge [bend left=30] (disvars2.east)
    edge [bend left=40] (disvars4.east)
    (goal.north) edge [bend right=20] (goal2.east);
    \path[arr]<5-> (disvars1.south) edge [bend left] (disvars2)
    edge [bend left] (disvars4);
  \end{tikzpicture}
\end{frame}


\begin{frame}
  \frametitle{Standard Semantics}
  \tikzstyle{every picture}+=[remember picture]

  $\As$: fixed model of the background theory
  \pause

  \vspace{0.5cm}
  \emph{standard} interpretation $\Sf$ of types: \hfill\tikz[baseline]\node[anchor=base] (sfunc1) {\color{lighterblue}\emph{full} function space}; 
  %\vspace*{-0.2cm}
  \begin{align*}
    \sinti\Sf\iota&\defeq\dom(\As)&\sinti\Sf o&\defeq\bool&\sinti\Sf{\tau\to\sigma}&\defeq
                                                                           \tikz[baseline]{
                                                                           \node[anchor=base] (sfunc2) {$[\sinti\Sf\tau\to\sinti\Sf\sigma]$};}
  \end{align*}

  \begin{tikzpicture}[overlay]
    \path[arr] (sfunc1.south) edge [bend right] (sfunc2);
  \end{tikzpicture}
  \pause

  \vspace{0.5cm}
  Structures $\Bs$, valuations $\alpha$ and denotations $\sinti\Bs
  M(\alpha)$ as usual\\
  \[
    \begin{array}{l}
      \text{e.g.}\qquad\sinti\Bs{M_1\,M_2}(\alpha)\defeq\sinti\Bs{M_1}(\alpha)(\sinti\Bs{M_2}(\alpha))
    \end{array}
  \]


\end{frame}

\begin{frame}
  \frametitle{HoCHC Satisfiability Problem}
  $\As$: fixed model (over $\Sigma$) of the background theory

  $\Set$: set of HoCHCs

  \vspace{3mm}
  
  {\large\begin{block}{Satisfiability}
      $\Set$ is \emph{$\As$-satisfiable} if there exists a $\Sigma'$-structure $\Bs$ s.t.
      \vspace{3pt}
      \begin{enumerate}
        \setlength\itemsep{5pt}
      \item $\Bs$ agrees with $\As$ on $\Sigma$ (background theory),\pause
      \item \emph{$\Bs,\alpha\models C$} for each $C\in\Set$ and valuation $\alpha$.
      \end{enumerate}
    \end{block}
  }
\end{frame}

\section{Canonical Model Property}
\partframe{Canonical Model Property}

\begin{frame}
  \frametitle{Immediate Consequence Operator}
  \tikzstyle{every picture}+=[remember picture]

  \onslide<2->{\subheadline{Idea:} \textit{satisfy what needs to be satisfied}}
  
  \vspace*{6mm}
  \onslide<3->{\hfill \tikz[baseline]\node[ann] (body1) {$\neg A_1\lor\cdots\lor\neg A_n\lor R\,\overline x\in\Set$};}
  %\vspace*{-1mm}
  {
    \large
    \begin{align*}
      {\color{oxred2}T_\Set(\Bs)}\onslide<3->{,\alpha\models R\,\overline x
      \hspace{5mm}&\iff\hspace{5mm}}
                    \Bs\onslide<3->{,\alpha\not\models
                    \tikz[baseline]{\node [e,inner sep=0pt] (body2)
                    {\emph{$\neg A_1\lor\cdots\lor\neg A_n$}};}
                    \hspace{10mm}
                    }
    \end{align*}
  }\pause\pause
  \begin{tikzpicture}[overlay]
    \path[->,arr] (body1.south) edge [bend right] (body2.north); 
  \end{tikzpicture}

  \pause
  \vspace*{-4mm}
  
  {\color{lighterblue}
    \Large\begin{align*}
            \text{prefixed points of }T_\Set = \text{models of \emph{definite} clauses in }\Set
          \end{align*}
        }

\end{frame}

\begin{frame}

  \vspace*{1cm}
  \onslide<2->{\large\color{lighterblue}\textbf{The \emph{least} model property
      \emph{fails} for standard semantics!}
  }

  \vspace{5mm}
  \begin{minipage}{0.5\textwidth}   %left column
    \subheadline{1st-order:}\\[3mm]
    {
      \centering 
      \begin{tikzcd}[row sep=2.5em,column sep=3.7em,minimum
        width=2em,nodes={anchor=center}, ampersand
        replacement=\&]
        T_\Set \text{ is monotone}
        \arrow[Rightarrow]{d}{\;\text{Knaster-Tarski}}\\
        \Set \text{ has \emph{least} model}\&
      \end{tikzcd}
    }
  \end{minipage}
  % \hfill\vline\hfill
  \begin{minipage}{0.45\textwidth} %right column
    \onslide<3->{
      \subheadline{higher-order:}\\[3mm]
      {\centering 
        \begin{tikzcd}[row sep=2.5em,column sep=3.7em,minimum
          width=2em,nodes={anchor=center}, ampersand
          replacement=\&]
          T_\Set \text{ is \emph{quasi-monotone}}
          \arrow[Rightarrow]{d}{\;\text{\parbox{2cm}{\emph{extension}
                of\\Knaster-Tarski}}}\\
          \Set \text{ has \emph{canonical} model}
        \end{tikzcd}
      }
    }
  \end{minipage}
\end{frame}


  \begin{frame}

    \subheadline{Fix:} $(L,\leq)$ complete lattice,
    \quad
    $F\from L\to L$,
    \quad
    \onslide<4->{\emph{${\arel}\subseteq L\times L$}}

    \vspace*{1mm}
    \onslide<2->{
      \begin{align*}
        a_0&\defeq\bot\qquad
             a_1\defeq F(a_0)\qquad
             a_2\defeq F(a_1)\quad\ldots\quad
             a_\omega\defeq \biglor_{n\in\omega} a_n\quad\ldots\\
      a_F&\defeq \biglor_{\beta\in\On}a_\beta
      \end{align*}
    }
    
    \vspace*{2mm}
    \onslide<5->{
      \begin{definition}
        $F$ is \emph{quasi-monotone} if $a\arel b\implies F(a)\arel F(b)$.
      \end{definition}
    }
    \onslide<3->{
      \begin{proposition}[``Extended Knaster-Tarski'']
        \begin{enumerate}
        \item $F(a_F)\leq a_F$
          \onslide<6->{
          \item
            
            $
            % \begin{align*}
            \left. \begin{array}{rl}
                     \text{\color{lighterblue}(i)}& F(b)\leq b\\
                     \text{\color{lighterblue}(ii)}& F \text{
                                                    is}
                                                    \tikz[baseline]{\node[e] (qm)
                                                    {quasi-monotone}}\\
                     \text{\color{lighterblue}(iii)}& \arel \text{ is
                                                      \emph{compatible} with } \leq \\
                   \end{array}\right\} \implies \emphmath{a_F\arel b}
                 % \end{align*}
                 $
               }
             \end{enumerate}
           \end{proposition}
         }
       \end{frame}


  \begin{frame}

    \tikzstyle{every picture}+=[remember picture]
    \vspace{3mm}
    \subheadline{Use:} $T_\Set$\onslide<3->{ and \emph{logical relations} ${\arel_\sigma}\subseteq\sinti\Sf\sigma\times\sinti\Sf\sigma$}
    
    \vspace*{4mm}
    \onslide<4->{
      \begin{lemma}[Fundamental Theorem]
        \vspace*{-7mm}
        \begin{align*}
          \left. \begin{array}{rl}
                   \Bs\arel\Bs'\\
                   \alpha\arel\alpha'\\
                 \end{array}\right\} \implies
          \sinti\Bs M(\alpha)\arel\sinti{\Bs'}M(\alpha')
        \end{align*}
        \vspace*{-4mm}
      \end{lemma}
    }
    \vspace*{4mm}
    {
      \setbeamertemplate{itemize item}[triangle]
      \setbeamercolor{itemize item}{fg=oxblue}
      \begin{itemize}
        \setlength\itemsep{1mm}
      \item<2-> \tikz[baseline]{\node[e] (cann1)
          {$\As_\Set$};}\hspace*{-1mm}$\models\{D\in\Set\mid D\text{
          definite}\}$\qquad\qquad\tikz[baseline]{\node[ann]
          (cann2) {canonical structure};}
      \item<5-> $T_\Set$ is quasi-monotone
      \item<5-> $\As_\Set\arel\Bs$ if $\Bs\models\Set$
      \item<6-> $\As_\Set\models$\hspace*{-2mm} \tikz[baseline]{\node[e] (goal1) {$G$};}\hspace*{-1mm}
        if $\Bs\models\Set$
        \qquad\qquad\quad\qquad\tikz[baseline]{\node[ann] (goal) {$G\in\Set$ goal clause};}
      \end{itemize}
    }
    

    \vspace*{4mm}
    \onslide<7->{
      \begin{theorem}[Canonical Model Property]
        $\As_\Set\models\Set$ if $\Set$ is $\As$-satisfiable.
      \end{theorem}
    }
    \onslide<6->{
      \begin{tikzpicture}[overlay]
        \path[arr] (goal.west) edge [bend left=20] ($ (goal1) + (0.2,-0.2) $);
      \end{tikzpicture}
    }
    \onslide<2->{
      \begin{tikzpicture}[overlay]
        \path[arr] (cann2.west) edge [bend right=20] ($ (cann1) + (0.2,0.2) $);
      \end{tikzpicture}
    }
    
  \end{frame}

\section{Resolution Proof System}
\partframe{Resolution Proof System}

\subsection{Proof System}
\begin{frame}
  %\frametitle{Proof System}

  \tikzstyle{every picture}+=[remember picture]

  \vspace*{2mm}
  \begin{block}{Proof System}
    \inferbintalk{Resolution}{G\lor \color{oxred2}\neg R\,\overline M}{{\color{oxred2}R\,\overline x}\lor G'}{G\lor \big(G'[\overline M/\overline x]\big)}
    \bigskip
    \pause

    \onslide<2->{
      \onslide<3->{\hfill\hfill \tikz\node[ann] (vars) {variables};} \hfill\tikz\node[ann] (bgs) {background atoms};

      \inferuntalk{Constraint\\Refutation}{
        \onslide<3->{\neg \tikz[baseline]{
          \node[anchor=base] (var1) {$x_1$};} \overline M_1\lor\cdots\lor
        \neg \tikz[baseline]{\node[anchor=base] (var2) {$x_m$};}\overline
        M_m\lor}
      \color{oxred2}\neg \tikz[baseline]{\node[anchor=base] (bg1) {$\phi_1$};}\lor\cdots\lor\neg \tikz[baseline]{\node[anchor=base] (bg2) {$\phi_n$};}}{\bot}{provided
      there exists a valuation $\alpha$ s.t.\ \emph{$\As,\alpha\not\models\neg\phi_1\lor\cdots\lor\neg\phi_n$}
    }
  }
  \end{block}

  \begin{tikzpicture}[overlay]
    % \node<2-> [rotate=-30] at (9.5,4.8) {\color{lighterblue}\footnotesize\parbox{3cm}{\centering(modulo renaming\\of variables)}};
    \path<3->[arr] (vars.west) edge [bend right=40] (var1)
    (vars.south) edge [bend left=10] (var2);
    \path<2->[arr](bgs.west) edge [bend right=40] (bg1)
    (bgs.south) edge [bend left=10] (bg2);
  \end{tikzpicture}
  \pause\pause
  {\footnotesize
    \hfill (+ rule for $\beta$-reduction in paper)}
        
\end{frame}

\subsection{Completeness}
\begin{frame}[label=outline]
  
  \tikzstyle{every picture}+=[remember picture]
  
  \vspace*{1.1cm}
    \begin{figure}
      \tikzcdset{every label/.append style = {font = \normalsize,outer
          sep=4pt}}

      \centering
      
      \vspace*{-8mm}
        
        % {\color{oxbrown}\large $\neg x\,M\lor\neg\phi\models\neg\phi$}

        % \vspace{0.6cm}
      
      \begin{tikzcd}[row sep=2.5em,column sep=3.7em,minimum width=2em,nodes={anchor=center}, ampersand replacement=\&]
        \tikz[baseline]{\node[box] (unsat) {\Large $\Set$ is $\As$-unsatisfiable};}
        \arrow[Rightarrow,visible on=<3->]{d}{}
        \arrow[Rightarrow,bend right=5,thick,dashed,visible on=<2-6>]{r}[below]{\textbf{completeness}}
        \arrow[Rightarrow,bend right=5,thick,visible on=<7->]{r}[below]{\textbf{completeness}}\&
        \tikz[baseline]{\node[box] (der)  {\Large $\bot$ is derivable
            from $\Set$};}
        \arrow[Rightarrow,bend right=5,thick]{l}[above]{\textbf{soundness}}\\
        \text{\onslide<3->{\parbox{3.9cm}{\large
              \textbf{\color{oxblue}1.} $\exists G\in\Set$ s.t.\\ 
              \hspace*{2cm}$\As_\Set\not\models G$}}}
        \arrow[Rightarrow,visible on=<5->]{d}{}\&
    \text{\onslide<7->{\parbox{5cm}{\large \textbf{\color{oxblue}4.}
          Sufficient to ``unfold'' at\\
          \emph{leftmost} positions of atoms}}}
        \arrow[Rightarrow,visible on=<7->]{u}{}\\
        \text{\onslide<5->{\parbox{3.9cm}{\large
              \textbf{\color{oxblue}2.} $\exists$
              \emph{$n\in\omega$} s.t.\\
              \hspace*{2cm}\tikz[baseline]{\node [e,inner sep=0pt] (cann)
                {\emph{$\As_n\not\models G$}};}}}}
        \arrow[Rightarrow,visible on=<6->]{r}{}\&
        \text{\onslide<6->{\parbox{5cm}{\textbf{\large
          \color{oxblue}3.} There is a \emph{syntactic} reason\\
        for $\As_n\not\models G$\\
        (by ``\emph{unfolding definitions}'')}}}
  \arrow[Rightarrow,visible on=<7->]{u}{}\\
  \tikz[baseline]{\onslide<4->{\node[box] (anncann)
    {\large $T_\Set$ is \emph{quasi-continuous}};}}
      \end{tikzcd}
    \end{figure}

    \begin{tikzpicture}[overlay]      
      \node [above left=0.4cm of der] {\LARGE\cmark};

      \node<5-> [above right=13mm and -15mm of anncann] (foo) {};


      \path[->,arr]<5-> (anncann.north) edge[bend right=20] (foo);
    \end{tikzpicture}
    
\end{frame}


% \begin{frame}[t]
%   \centering

%   {\color{oxbrown}\large $\neg x\,M\lor\neg\phi\models\neg\phi$}


%   \begin{tikzpicture}[ampersand replacement=\&]
%     \matrix (m) [matrix of nodes,row sep=3em,column sep=5em,minimum width=2em,nodes={anchor=center}]
%     {
%       \node[box] (unsat) {\Large $S$ is $\As$-unsatisfiable}; \&
%       \node[box] (der)  {\Large $\bot$ is derivable
%         from $S$}; \\
%       \node<3->[proof] (cont) [text width=3.9cm] {\large
%         \textbf{\color{oxblue}1.} $S$ is
%         $\As$-\emph{continuous}-unsatisfiable};\&
%       \node<6->[proof] (left) [text width=5cm] {\large \textbf{\color{oxblue}4.}
%         Sufficient to \textit{unfold} at \emph{leftmost} positions of
%         atoms
%       };\\
%       \node<4->[proof] (cann) [text width=3.9cm] {\large
%         \textbf{\color{oxblue}2.} ex. $G\in S$ and $n\in\omega$ s.t.\
%         \emph{$\As_n^\Cf\not\models G$}};\&
%       \node<5->[proof] (syntax) [text width=5cm] {\large
%         \textbf{\color{oxblue}3.} Reason for $\As_n^\Cf\not\models G$ can be captured
%         \emph{syntactically} by \emph{unfolding definitions}}; \\
%     };

%     \onslide<2->{\node (scheck) [above left=0.4cm of der] {\LARGE\cmark};}
%     \node<4->[ann] (anncann) [below right=0.8cm and -1.2cm of cann, text width=4cm]
%     {nth-stage of continuous
%       \emph{canonical} \color{lighterblue}structure}; 

%     \path[->,thick] (der) edge[above,double,bend right=5] node {\large soundness}  (unsat);
%     \path[->,thick]<2-> (unsat) edge[below,double,bend right=5] node {\large
%       completeness}  (der);

%     \path[->]<3-> (unsat) edge[double] (cont);
%     \path[->]<4-> (cont) edge[double] (cann);
%     \path[->]<5-> (cann) edge[double] (syntax);
%     \path[->]<6-> (syntax) edge[double] (left)
%     (left) edge[double] (der);

%     \path[->,arr]<4-> (anncann.west) edge[bend left=30] (cann.south);
%   \end{tikzpicture}
  
% \end{frame}



% \begin{frame}
%   \frametitle{Outline of the Completeness Proof}
%   {\large
%     \begin{enumerate}
%       \setlength\itemsep{1em}
%     \item $S$ is $\As$-\emph{continuous}-unsatisfiable if $S$ is
%       $\As$-unsatisfiable.
%     \item Iteratively construct a continuous \emph{canonical model} $\As^\Cf_\omega$ of
%       the definite clauses in $S$.
%       {
%         \setbeamertemplate{itemize subitem}[triangle]
%         \setbeamercolor{itemize subitem}{fg=oxblue}
%         \setbeamerfont{itemize/enumerate subbody}{size=\large}
%         \begin{itemize}
%         \item exists $G\in S$ and $n\in\omega$ s.t.\ $\As_n^\Cf\not\models
%           G$
%         \end{itemize}
%       }
%     \item Reason for $\As_n^\Cf\not\models G$ can be captured \emph{syntactically} by \emph{unfolding definitions}.
%     \item Sufficient to \textit{unfold} at \emph{leftmost} positions of
%       atoms
%       {
%         \setbeamertemplate{itemize subitem}[triangle]
%         \setbeamercolor{itemize subitem}{fg=oxblue}
%         \setbeamerfont{itemize/enumerate subbody}{size=\large}
%         \begin{itemize}
%         \item can be captured by the resolution proof system
%         \end{itemize}
%       }
%     \end{enumerate}
%   }
% \end{frame}



% \begin{frame}

%   {\LARGE
%     \begin{enumerate}
%     \item \color{lighterblue} $\Set$ is $\As$-\emph{continuous}-unsatisfiable if $\Set$ is
%       $\As$-unsatisfiable
%     \end{enumerate}
%   }
% \end{frame}

% \begin{frame}
%   \frametitle{Continuous Semantics}
%   \tikzstyle{every picture}+=[remember picture]

%    \emph{continuous} interpretation $\Cf$ of types: 
%   %\vspace*{-0.2cm}
%   \begin{align*}
%     \sinti\Cf\iota&\defeq\dom(\As)&\sinti\Cf o&\defeq\bool&\sinti\Cf{\tau\to\sigma}&\defeq
%                                                                            \color{oxred2}\tikz[baseline]{
%                                                                            \node[anchor=base] (cfunc2) {$[\sinti\Cf\tau\xrightarrow{c}\sinti\Cf\sigma]$};}
%   \end{align*}
% \hfill\tikz[baseline]\node[anchor=base] (cfunc1)
% {\color{lighterblue}\emph{continuous} function space};
% \begin{tikzpicture}[overlay]
%     \path[arr] (cfunc1.north) edge (cfunc2);
%   \end{tikzpicture}
% \pause
% \begin{definition}
% A monotone $f\from \tikz[baseline]{
%   \node[anchor=base] (con2) {$P$};} \to \tikz[baseline]{
%   \node[anchor=base] (con3) {$Q$};}$ is \emph{continuous} if for all
% \textit{directed}   $D\subseteq P$,
% \begin{align*}
%   f\left(\bigsqcup D\right)=\bigsqcup_{d\in D}f(d)
% \end{align*}

% \vspace{-0.7cm}\tikz[baseline]\node[anchor=base,ann] (con1)
%   {directed-complete posets}; 
% \end{definition}
% \begin{tikzpicture}[overlay]
%     \path[arr] (con1) edge [bend left] (con2)
%     (con1) edge [bend left=15] (con3);
%   \end{tikzpicture}\pause
  
%   Structures $\Bs$, valuations $\alpha$ and denotations $\sinti\Bs
%   M(\alpha)$ still as usual \textit{\color{oxbrown!90!black}(but w.r.t.\
%     $\sinti\Cf-$!)}
% \end{frame}

% \begin{frame}

%   \tikzstyle{every picture}+=[remember picture]

%   % \tikz[baseline]\node[anchor=base,ann] (constr1)
%   % {continuous structure};
%   % {\Large
%   %   \begin{align*}
%   %     \tikz[baseline]{\node[anchor=base] (constr2)
%   %     {$I(\Bs)$};},\tikz[baseline]{\node[anchor=base] (stval2)
%   %     {$\alpha$};}\not\models G\text{ implies }\tikz[baseline]{\node[anchor=base] (constr3)
%   %     {$\Bs$};},L\circ \tikz[baseline]{\node[anchor=base] (stval3)
%   %     {$\alpha$};}\not\models G
%   %   \end{align*}
%   % }
% % \hfill\tikz[baseline]\node[anchor=base,ann] (stval1)
% %   {standard valuation};

%   % \begin{tikzpicture}[overlay]
%   %   \draw [draw=lighterblue,decoration={brace, mirror,raise=0.5cm,
%   %     amplitude=5pt}, decorate] (1.3,1.9) -- (2.4,1.9) 
%   %   node [ann,pos=0.5,anchor=south,yshift=-1.3cm] {standard structure};

%   %   \draw [draw=lighterblue,decoration={brace, raise=0.5cm,
%   %     amplitude=5pt}, decorate] (7.0,1.8) -- (8.4,1.8) 
%   %   node [ann,pos=0.5,anchor=north,yshift=1.3cm] {continuous valuation};
  
    
%   %   \path[arr] (constr1.south) edge [bend right] (constr2.north)
%   %   (constr1.east) edge [bend left=20] (constr3)
%   %   (stval1.west) edge [bend left=20] (stval2)
%   %   (stval1.north) edge [bend left=10] (stval3);
%   % \end{tikzpicture}  \pause

%   {\large
%     \begin{theorem}
%       If $\Set$ is $\As$-continuous-satisfiable then it is $\As$-satisfiable.
%     \end{theorem}
%   }\pause
%   \textbf{Proof sketch.}
%     Define adjunctions for each type $\sigma$:
%     \vspace*{-2mm}
%   {\large
%     \begin{figure}
%     \centering
%     \begin{tikzcd}[ampersand replacement=\&]
%       \sinti\Cf\sigma \arrow [shift left=1ex]{r}[name=C]{I_\sigma}
%       \arrow[draw=none, shift left=-1ex]{r}{\top}\&
%       \sinti\Sf\sigma\arrow[shift left=1ex]{l}[name=S]{L_\sigma}
%     \end{tikzcd}      
%     \end{figure}
%   }
%   \vspace*{-2mm}
%   $\color{lighterblue}\underbrace{\color{black}I(\Bs),\alpha\not\models G}_{\color{lighterblue}\textstyle\text{standard}}$ implies $\color{lighterblue}\underbrace{\color{black}\Bs,L\circ\alpha\not\models G}_{\color{lighterblue}\textstyle\text{continuous}}$

%   \vspace{-4mm}\qed
  
% \end{frame}

% \begin{frame}

%   \tikzstyle{every picture}+=[remember picture]
%   {\LARGE
%     \begin{enumerate}
%     \item[2.] \color{lighterblue} $\exists G\in\Set$ and \emph{$n\in\omega$} s.t.\
%       \emph{$\tikz[baseline]{\node[anchor=base] (cann)
%       {$\As_n^\Cf$};}\not\models G$}
%     \end{enumerate}
%   }

%   \vspace{0.7cm}
%   \hfill\tikz\node[ann] (anncann) [text width=5cm]
%   {\Large nth-stage of continuous
%     \emph{canonical} \color{lighterblue}structure}; 

%  \begin{tikzpicture}[overlay]
%    \path[->,arr] (anncann) edge[bend left=30] (cann);
%   \end{tikzpicture}
  
% \end{frame}


% {\againframe<6>{outline}}

\section{Semantic Invariance}
\partframe{Semantic Invariance}


\subsection{Semantic Invariance}
\begin{frame}<1-2>[label=seminv]
  \frametitle{Semantic Invariance}

  \tikzstyle{every picture}+=[remember picture]
  
  \vspace*{15mm}
    \begin{figure}
      \tikzcdset{every label/.append style = {font = \normalsize,outer
          sep=4pt}}
      
      \begin{tikzcd}[column sep=2em,row sep=3em, ampersand replacement=\&,overlay]
        \tikz{\onslide<2->{\node[box] (mon)  {\Large $\As$-monotone-satisfiable};}}
        \arrow[Leftrightarrow,thick,visible on=<2->]{d}{\text{[POPL'18]}}\&
        \tikz[baseline]{\onslide<2->{\node[e]{$\sinti\Mf{\tau\to\sigma}\defeq[\sinti\Mf\tau$};}}
        \hspace{-0.2cm}\tikz[baseline]{\onslide<2->{\node[frame]{$\xrightarrow{m}$};}}
        \hspace{-0.2cm}\tikz[baseline]{\onslide<2->{\node[e]{$\sinti\Mf\sigma]$};}}\\
        \tikz\node[box] (std) {\Large \textbf{$\As$-satisfiable}};
        \arrow[Leftarrow,thick,visible
        on=<4>]{d}{\parbox{2.2cm}{soundness +\\completeness}}
        \arrow[Leftrightarrow,thick,visible on=<5->]{d}{\parbox{2.2cm}{soundness +\\completeness}}\&
        \sinti\Sf{\tau\to\sigma}\defeq[\sinti\Sf\tau\xrightarrow{}\sinti\Sf\sigma]\\
        \tikz{\onslide<3->{\node[box] (cont)  {\Large $(\As,\Hf)$-satisfiable};}}
        %\arrow[Rightarrow,thick,shift left=1.3ex]{u}{\text{\large\cmark}}
        \arrow[Rightarrow,thick,shift left=1.3ex,visible on=<7->]{d}{\text{\large\cmark}}\&
        \tikz[baseline]{\onslide<3->{\node[e] (subset) {$\sinti\Hf{\tau\to\sigma}$};}}
        \hspace{-0.2cm}\tikz[baseline]{\onslide<3->{\node[frame]{$\subseteq$};}}
        \hspace{-0.2cm}\tikz[baseline]{\onslide<3->{\node[e]{$[\sinti\Hf\tau\xrightarrow{}\sinti\Hf\sigma]$};}}\\
        \tikz{\onslide<6->{\node[box] (Henkin)  {\Large $\As$-Henkin-satisfiable};}}
        \arrow[Rightarrow,thick,shift left=1.3ex,visible
        on=<8->]{u}{\parbox{2.2cm}{soundness +\\completeness}}\&
        \tikz[baseline]{\onslide<6->{\node[e]{$(\As,\Hf')$-satisfiable for \emph{some} $\Hf'$};}}
      \end{tikzcd}
    \end{figure}

    \begin{tikzpicture}[overlay]
      \node[e,right=-10mm of subset] (foo) {}; 
      \node<5->[ann,below right=2mm and -4mm of subset] (closed) {closed under suprema};
      \path[->,arr]<5-> (closed.west) edge[bend left=40] (foo.south);
    \end{tikzpicture}
    
\end{frame}



\begin{frame}

  \color{black!80}
  \tikzstyle{every picture}+=[remember picture]
  \only<1-3>{
     \tikzstyle{fr2}=[anchor=base,ellipse,inner sep=1pt]
  }
  \only<4->{
    \tikzstyle{fr2}=[fill=oxpink,anchor=base,very
    thick,dashed,ellipse,inner sep=1pt]
  }

  \vspace*{1.1cm}
    \begin{figure}
      \tikzcdset{every label/.append style = {font = \normalsize,outer
          sep=4pt}}

      \centering
      
      \vspace*{-4mm}
        
        % {\color{oxbrown}\large $\neg x\,M\lor\neg\phi\models\neg\phi$}

        % \vspace{0.6cm}
      
      \begin{tikzcd}[row sep=2.5em,column sep=3.7em,minimum width=2em,nodes={anchor=center}, ampersand replacement=\&]
        \tikz[baseline]{\node[greybox] (unsat) {\Large $\Set$ is $\As$-unsatisfiable};}
        \arrow[Rightarrow,visible on=<3->]{d}{}
        \arrow[Rightarrow,bend right=5,thick,visible on=<3->]{r}[below]{\textbf{completeness}}\&
        \tikz[baseline]{\node[greybox] (der)  {\Large $\bot$ is derivable
            from $\Set$};}
        \arrow[Rightarrow,bend right=5,thick]{l}[above] {\textbf{soundness}}\\
        \text{\onslide<3->{\parbox{3.9cm}{\large
            \textbf{1.} $\exists G\in\Set$ s.t.\\ 
            \hspace*{1.7cm}\tikz[baseline]{\node[fr2] (cann2) {$\As_\Set$};}$\not\models G$}}}
        \arrow[Rightarrow,visible on=<3->]{d}{}\&
    \text{\onslide<3->{\parbox{5cm}{\large \textbf{4.}
          Sufficient to ``unfold'' at\\
          leftmost positions of atoms}}}
        \arrow[Rightarrow,visible on=<3->]{u}{}\\
        \text{\onslide<3->{\parbox{3.9cm}{\large
              \textbf{2.} $\exists$
              $n\in\omega$ s.t.\\
              \hspace*{1.7cm}
              $\As_n\not\models G$}}}
        \arrow[Rightarrow,visible on=<3->]{r}{}\&
        \text{\onslide<3->{\parbox{5cm}{\textbf{\large
          3.} There is a syntactic reason\\
        for $\As_n\not\models G$\\
        (by ``unfolding definitions'')}}}
    \arrow[Rightarrow,visible on=<3->]{u}{}
  \end{tikzcd}
    \end{figure}

    \begin{tikzpicture}[overlay]
      %12,53
      \node [rotate=-30,above right=12mm and -13mm of der,draw=black!75] {\Large\textbf{Recap}};

      \onslide<4->{\node[below right=57mm and -5mm of unsat,highlight] (sup) {\parbox{30mm}{$\sinti\Sf\rho$ closed
            under suprema}};}

    \onslide<2->{\node [above left=13mm and 9mm of der,highlight] (exp) {\parbox{33mm}{denotation of terms as
            ``expected''}};
      \node [below=4mm of exp] (sound) {};}


   \node [above right=31mm and -20mm of anncann] (bar) {};

     \path[harr]<4-> (sup.north) edge[bend right=20] (bar);
     \path[harr]<2-> (exp) edge[] (sound);


    \end{tikzpicture}
    
  \end{frame}

  \begin{frame}
    \frametitle{Generalised HoCHC Satisfiability Problem}
    \tikzstyle{every picture}+=[remember picture]

    $\As$: fixed model of the background theory

    $\Set$: set of HoCHCs

    $\Hf$: fixed interpretation of types s.t.\pause
    \hfill\tikz[baseline]\node[anchor=base] (sfunc1)
    {\parbox{40mm}{\color{lighterblue}sufficiently rich to give ``expected'' denotations}}; 
    % \vspace*{-0.2cm}
    \begin{align*}
      \sinti\Hf\iota&=\dom(\As)&\sinti\Hf
                                 o&=\bool&\sinti\Hf{\tau\to\sigma}&
                                                                    \tikz[baseline]{\node[frame]
                                                                    (subset)
                                                                    {$\subseteq$};}
                                                                    \tikz[baseline]{
                                                                    \node[anchor=base] (sfunc2) {$[\sinti\Hf\tau\to\sinti\Hf\sigma]$};}
    \end{align*}

    \begin{tikzpicture}[overlay]
      \path[arr] (sfunc1.south) edge [bend left=10] (subset.north);
    \end{tikzpicture}
    \pause
    \begin{block}{Satisfiability}
      $\Set$ is \emph{$(\As,\Hf)$-satisfiable} if there exists a \emph{$(\Sigma',\Hf)$-structure} $\Bs$ s.t.\pause
      \vspace{3pt}
      \begin{enumerate}
        \setlength\itemsep{5pt}
      \item $\Bs$ agrees with $\As$ on $\Sigma$ (background theory),
      \item $\Bs\models\Set$.
      \end{enumerate}
    \end{block}

  \end{frame}

  \againframe<3->{seminv}




\section{Conclusion}
\begin{frame}
  \frametitle{Conclusion}


  
  \subheadline{This talk:}
  \begin{itemize}
  \item A \emph{simple} resolution proof system for HoCHC
    \begin{itemize}
    \item Completeness even for \emph{standard} semantics
    \end{itemize}
  \item Canonical model property and semantic invariance of HoCHC
  \end{itemize}
  \pause
  \subheadline{Also in the paper:}
  \begin{itemize}
  \item Extension to \emph{compact} theories
  \item 1st-order translation (complete for \emph{standard}
    semantics)
  \item \emph{Decidable} fragments
  \end{itemize}
  \pause
  \subheadline{Future directions:}
  \begin{itemize}
  \item Implementation
  \item Improve \textit{robustness} on satisfiable instances
  \end{itemize}
  \vspace{15pt}
%   \vfill
% {  \footnotesize\emph{Horus}
%   (\url{http://mjolnir.cs.ox.ac.uk/horus/})\\[-3pt]
%   \emph{DefMono} (\url{http://mjolnir.cs.ox.ac.uk/dfhochc/})}
\end{frame}

\setbeamertemplate{footline}{%
 \includegraphics[align=c,height=7mm]{logo2.png}%
  \hspace*{-3cm}
  \usebeamercolor[bg]{footer}
  \usebeamercolor[fg]{footer}
  \hfill
  \insertframenumber{}/\inserttotalframenumber\,
}


\begin{frame}
  \frametitle{Conclusion}

  \vspace*{1cm}
  
  \begin{columns}
    \begin{column}{0.8\paperwidth}
      \textit{\textbf{\Large\color{lighterblue} 
    HoCHC lies at a \textbf{``sweet
      spot''} in higher-order logic, \emph{semantically robust} and
    \emph{useful} for \emph{algorithmic} verification.}}

    \end{column}
  \end{columns}

    
  {          \centering
          % \includegraphics[align=c, height=7mm]{../logo2.png}%

          \vspace*{15mm}
          
          \insertauthor
          
          \vspace*{3mm}
          
          \href{mailto: dominik.wagner@cs.ox.ac.uk}{\texttt{dominik.wagner@cs.ox.ac.uk}}
        % \begin{mdframed}[backgroundcolor=lighterblue,fontcolor=white,
      %     userdefinedwidth=\paperwidth,
      %     innertopmargin=10pt,innerbottommargin=10pt,
      %     innerleftmargin=2pt,innerrightmargin=2pt]
      
      %     \centering
      %     \stepcounter{partc}
      %     \Huge Part \Roman{partc}:

      %     \textbf{#1}
      %   \end{mdframed} 
      % \
        }
        
\end{frame}



\appendix
% \begin{frame}
%         \frametitle{References}
%         \bibliographystyle{apalike}
%         \bibliography{lit}
% \end{frame}

\subsection{Example}
\begin{frame}[noframenumbering]

  \only<2-3>{\tikzstyle{res1}=[fill=oxpink,inner sep=3pt,anchor=base]}
  \only<1,4->{\tikzstyle{res1}=[inner sep=3pt,anchor=base]}
  \only<3>{\tikzstyle{add1}=[fill=oxgreen3,inner sep=3pt,anchor=base]}
  \only<-2,4->{\tikzstyle{add1}=[inner sep=3pt,anchor=base]}

  \only<3>{\tikzstyle{special1}=[fill=oxgreen3,inner sep=3pt,anchor=base]}
  \only<4-5>{\tikzstyle{special1}=[fill=oxpink,inner sep=3pt,anchor=base]}
  \only<-2,6->{\tikzstyle{special1}=[inner sep=3pt,anchor=base]}
  
  
  \only<4-5>{\tikzstyle{res2}=[fill=oxpink,inner sep=3pt,anchor=base]}
  \only<-3,6->{\tikzstyle{res2}=[inner sep=3pt,anchor=base]}
  \only<5>{\tikzstyle{add2}=[fill=oxgreen3,inner sep=3pt,anchor=base]}
  \only<-4,6->{\tikzstyle{add2}=[inner sep=3pt,anchor=base]}

  \only<6-7>{\tikzstyle{res3}=[fill=oxpink,inner sep=3pt,anchor=base]}
  \only<-5,8->{\tikzstyle{res3}=[inner sep=3pt,anchor=base]}
  \only<7>{\tikzstyle{add3}=[fill=oxgreen3,inner sep=3pt,anchor=base]}
  \only<-6,8->{\tikzstyle{add3}=[inner sep=3pt,anchor=base]}


  \only<7>{\tikzstyle{special3}=[fill=oxgreen3,inner sep=3pt,anchor=base]}
  \only<8-9>{\tikzstyle{special3}=[pattern=crosshatch,pattern color=oxpink,inner sep=3pt,anchor=base]}
  \only<-6>{\tikzstyle{special3}=[inner sep=3pt,anchor=base]}
  
  \only<8-9>{\tikzstyle{res4}=[pattern=crosshatch,pattern color=oxpink, inner sep=3pt,anchor=base]}
  \only<-7>{\tikzstyle{res4}=[inner sep=3pt,anchor=base]}
  \only<9>{\tikzstyle{add4}=[fill=oxgreen3,inner sep=3pt,anchor=base]}
  \only<-8>{\tikzstyle{add4}=[inner sep=3pt,anchor=base]}

  
  \vspace*{0.4cm}
  \begin{mdframed}[backgroundcolor=lightblue,linecolor=oxblue,linewidth=1pt]
    \vspace*{-0.5cm}
    \small
    \begin{align*}
      \tikz[baseline]{\node[add2] {$\neg(z=x+y)$};}\lor
      \tikz[baseline]{\node[res2] {$\Add\,x\,y\,z$};}&=:D_1\\[-7pt]
      \tikz[baseline]{\node[add3] {$\neg(n\leq 0)$};}\lor
      \tikz[baseline]{\node[add3] {$\neg (s=x)$};} \lor
      \tikz[baseline]{\node[res3] {$\Iter\,f\,s\,n\,x$};}&=:D_2\\[-7pt]
      \tikz[baseline]{\node[add1] {$\neg(n>0)$};}\lor
      \tikz[baseline]{\node[add1] {$\neg \Iter\,f\,s\,(n-1)\,y$};}\lor
      \tikz[baseline]{\node[add1] {$\neg(f\,n\,y\,x)$};} \lor
      \tikz[baseline]{\node[res1] {$\Iter\,f\,s\,n\,x$};}&=:D_3
    \end{align*} 
  \end{mdframed}
  \vspace*{0.3cm}
  
  \begin{prooftree}
    \AxiomC{$\neg(n\geq 1)\lor\tikz[baseline]{\node[res1] {$\neg\Iter\,\Add\,n\,n\,x$};}\lor\neg(x\leq
      n+n)$}
    \AxiomC{\onslide<3->{\hspace*{-0.2cm}$D_3$}}
    \LeftLabel{\onslide<3->{\color{oxblue}Res.}}
    \only<-2>{\noLine}
    \BinaryInfC{\onslide<3->{\parbox{7.7cm}{
        \raggedleft $\neg(n\geq 1)\lor
        \tikz[baseline]{\node[add1] {$\neg(n>0)$};}\lor
        \tikz[baseline]{\node[add1] {$\neg\Iter\,\Add\,n\,(n-1)\,y$};}\lor$\\
        $\tikz[baseline]{\node[special1] {$\neg \Add\,n\,y\,x$};}\lor\neg(x\leq n+n)$
      }}}
    \AxiomC{\onslide<5->{\hspace*{-0.2cm}$D_1$}}
    \LeftLabel{\onslide<5->{\color{oxblue}Res.}}
     \only<-4>{\noLine}
     \BinaryInfC{\onslide<5->{\parbox{7.5cm}{\raggedleft $\neg(n\geq
        1)\lor\neg(n>0)\lor
        \tikz[baseline]{\node[res3] {$\neg\Iter\,\Add\,n\,(n-1)\,y$};}\lor$\\
        $\tikz[baseline]{\node[add2] {$\neg (x=n+y)$};}
        \lor\neg(x\leq n+n)$}}}
    \AxiomC{\onslide<7->{\hspace*{-0.2cm}$D_2$}}
    \LeftLabel{\onslide<7->{\color{oxblue}Res.}}
    \only<-6>{\noLine}
    \BinaryInfC{\onslide<7->{\parbox{7.0cm}{\raggedleft
          $\tikz[baseline]{\node[res4] {$\neg(n\geq 1)$};}\lor
          \tikz[baseline]{\node[res4] {$\neg(n>0)$};}\lor
          \tikz[baseline]{\node[special3] {$\neg (n-1\leq 0)$};}\lor$\\
          $\tikz[baseline]{\node[special3] {$\neg(n=y)$};}\lor
          \tikz[baseline]{\node[res4] {$\neg(x=n+y)$};}\lor
          \tikz[baseline]{\node[res4] {$\neg(x\leq n+n)$};}$}}}
    \LeftLabel{\onslide<9->{\color{oxblue}Const.\ Ref.}}
    \only<-8>{\noLine}
    \UnaryInfC{\onslide<9->{\tikz[baseline]{\node[add4] {$\bot$};}}}   
  \end{prooftree}
  \onslide<8->{
    \begin{align*}
      \alpha(n)&=1&\alpha(x)&=2&\alpha(y)&=1
    \end{align*}
}
\end{frame}


\end{document}

